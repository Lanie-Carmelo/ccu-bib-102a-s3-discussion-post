% filepath: /home/lanie/ccu/apa-7-student-paper-template/main.tex
% APA 7 Student Paper Template
% Author: Lanie Molinar Carmelo
% Repository: https://github.com/Lanie-Carmelo/APA-7-Student-Paper-Template
% Template Version: 1.2.0 (2025-10-20)
% License: MIT License - https://opensource.org/licenses/MIT
%
% ATTRIBUTION: If you use this template, please consider:
%   - Citing the repository in your acknowledgments or documentation
%   - Keeping this header intact to help others find the source
%   - Checking for updates at the repository URL above
%
% Usage: Fill in metadata, write content, and build with 'make pdf'
% NOTE: This template uses biblatex with the biber backend.
%       Compile with LuaLaTeX for best results and font support.
%       If building manually, run:
%         lualatex main.tex
%         biber main
%         lualatex main.tex
%         lualatex main.tex

\documentclass[12pt,letterpaper]{article}

% Layout and spacing
\usepackage[margin=1in]{geometry}
\usepackage{setspace}
\doublespacing
\usepackage{microtype} % Improves typography
\usepackage{url}
\urlstyle{same}

% Allow slightly looser line breaking to avoid overfull boxes
\tolerance=1000
\emergencystretch=2em

% Font and encoding
\usepackage{fontspec}
\setmainfont{Times New Roman}

% Header (APA-style)
\usepackage{fancyhdr}
\setlength{\headheight}{14.5pt}
\pagestyle{fancy}
\fancyhf{}
\fancyhead[L]{Session 3 Discussion}  % No "RUNNING HEAD:" for student papers
\fancyhead[R]{\thepage}
\renewcommand{\headrulewidth}{0pt} % Remove header line per APA

% Section formatting (APA: no numbering)
\setcounter{secnumdepth}{0}

% Language and citation setup
\usepackage[american]{babel}
\usepackage{csquotes}
\usepackage[style=apa,backend=biber,sortcites=true,sorting=nyt]{biblatex}
\DeclareLanguageMapping{american}{american-apa}
\addbibresource{references.bib}

% Accessibility and metadata
\usepackage{hyperref}
\hypersetup{
  pdftitle={Session 3 Discussion Post},
  pdfauthor={Lanie Molinar},
  pdfsubject={New Testament Introduction (BIB-102A)},
  pdfkeywords={APA 7, Student Paper, LaTeX, Lanie Molinar, BIB-102A, CCU, New Testament Introduction},
  pdflang={en-US},
  colorlinks=true,
  linkcolor=blue,
  citecolor=blue,
  urlcolor=blue,
  bookmarksnumbered=true,
  pdfstartview=FitH
}

% Optional: Silence biblatex footnote warning
\usepackage{silence}
\WarningFilter{biblatex}{Patching footnotes failed}

% APA title page customization
\usepackage{titling}
\pretitle{\begin{center}\large\bfseries}
\posttitle{\end{center}}

\title{Session 3 Discussion Post}
\author{Lanie Molinar\\Colorado Christian University}
\date{October 22, 2025}

\begin{document}

% Title page
\begin{titlepage}
\maketitle
\thispagestyle{fancy}

\begin{center}
New Testament Introduction (BIB-102A)\\
Instructor: Jeff Augustine
\end{center}
\end{titlepage}

% Main content starts on page 2
\section{Discussion Post}

The conversion stories of Cornelius \parencite[Acts 10–11]{Tyndale1996} and Saul \parencite[Acts 9; 22; 26]{Tyndale1996} reveal how the gospel transcends cultural boundaries and radically transforms individuals. In both cases, the death and resurrection of Jesus Christ are central to the apostles’ message. Peter tells Cornelius that Jesus was crucified and raised to life, and that forgiveness is available to all who believe in Him \parencite[Acts 10:39–43]{Tyndale1996}. Similarly, Saul encounters the risen Christ directly, which becomes the foundation of his entire ministry.

In both stories, the apostles emphasize that salvation comes through faith in Jesus and obedience to His call. Peter tells Cornelius that “everyone who believes in him will have their sins forgiven through his name” \parencite[Acts 10:43]{Tyndale1996}. Saul is told to “be baptized and wash away your sins, calling on his name” \parencite[Acts 22:16]{Tyndale1996}.

The Holy Spirit plays a vital role in both conversions. Cornelius and his household receive the Spirit before baptism, affirming that God accepts Gentiles without distinction \parencite[Acts 10:44–48]{Tyndale1996}. Many Jews tried to claim that gentiles needed to do specific things to be saved \parencite[chapter 15]{elwellEncounteringNewTestament2022}, but these conversions challenge that view and affirm that salvation is by faith, not ritual. Saul is filled with the Spirit after his baptism, empowering him to preach boldly \parencite[Acts 9:17–20]{Tyndale1996}.

Other passages like \textcite[John 3:5]{Tyndale1996}, \textcite[Romans 8:9–11]{Tyndale1996}, and \textcite[Titus 3:5–6]{Tyndale1996} clarify that the Holy Spirit is essential for regeneration, guidance, and assurance of salvation.

These stories remind me that salvation is not limited by background, ability, culture, or past mistakes. Whether a Roman centurion or a persecutor of Christians, anyone can be transformed by the grace of Jesus and the power of the Spirit. They challenge me to share the gospel boldly and inclusively, trusting that God can work in anyone’s life.

These stories resonate with my own coming to faith in profound ways. I grew up as a Christian, but because I only knew the King James Version of the Bible, which I struggled to understand, my faith never felt as strong as it could be. I learned about Jesus mostly through Christian television programs and music. In 2014, I went through a period of deep isolation and pain. My caregiver, my mom, was temporarily out of the picture, and I was living with untreated fibromyalgia, alone and unable to leave an upstairs apartment. During that time, I began searching for a church and a Bible I could understand. That’s when missionaries from the Church of Jesus Christ of Latter-day Saints knocked on my door. They offered support and seemed to have answers when I felt most vulnerable, and I quickly joined. But over time, I became increasingly unhappy. I was pressured to tithe despite financial hardship, to attend temple services that were physically exhausting, and to meet expectations that didn’t align with my health or understanding of grace. With my mom’s help and growing spiritual discomfort, I made the difficult decision to leave. That choice cost me relationships, including someone who had called me her adopted granddaughter, but it also opened the door to healing. Like Saul, I had to walk away from something that once felt right in order to truly follow Jesus. Through my current church and my studies at CCU, I’m learning and growing in faith daily, just as Cornelius and Saul did after their encounters with the gospel.

\newpage

\printbibliography[heading=bibintoc]

\end{document}
